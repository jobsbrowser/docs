\documentclass[12pt,twoside,a4paper]{article}
% ten dokument należy kompilować za pomocą XeLaTeX-a. Trzeba mieć zainstalowane w systemie czcionki Arial i Arial Narrow

\usepackage[utf8]{inputenc}
\usepackage{fontspec,newunicodechar}
\defaultfontfeatures{Ligatures=TeX}
\usepackage[small,sf,bf]{titlesec}

\usepackage{indentfirst}
\setlength{\parindent}{5mm}

\usepackage{hyphenat}
\hyphenation{in-for-ma-cyj-nych}

\usepackage[MeX]{polski}
\usepackage[T1]{fontenc}
\usepackage{graphicx}
\usepackage{geometry}
\usepackage{a4wide}
\geometry{left=20mm,right=20mm,bindingoffset=10mm, top=25mm, bottom=25mm}



\newfontfamily\arial{Arial}

\linespread{1.5}

\usepackage{xunicode}
\usepackage{xltxtra}
 
 
\usepackage{amsfonts}
\usepackage{amsmath,amssymb,amsthm}
\usepackage{latexsym,xpatch}
\usepackage{mathrsfs}
\usepackage{enumerate}
\usepackage{verbatim}
\usepackage{textcomp}
\usepackage{multirow}


\def\discipline{Informatyka} % lub Informatyka
\def\spec{Matematyka w naukach informacyjnych}
\def\title{Automatyczna analiza rynku pracy IT}

\def\authori{Bartłomiej Sielicki}
\def\albumi{268807}
\def\authorii{Łukasz Skarżyński}
\def\albumii{268808}

\def\supervisor{mgr Barbara Rychalska}
\def\year{2018}

\begin{document}
\sloppy
\pagestyle{empty}


\includegraphics[scale=1.]{politechnika} 

\begin{center}
\vspace{40pt}

\includegraphics[scale=1.]{praca_inz}  % lub praca_lic lub praca_mgr

{ \arial na kierunku \discipline

\vspace{30pt}
{\arial \large \title}

\vspace{40pt}

{\arial \huge \authori }

\vspace{5pt}

Numer albumu \albumi

\vspace {20pt}
{\arial \huge \authorii}

\vspace{5pt}

Numer albumu \albumii

\vspace{40pt}

promotor \\
{\arial \supervisor}

\vspace{15pt}
 
konsultacje  \\
{\arial dr hab. inż. Przemysław Biecek }

 \vfill
WARSZAWA \year \\
}
\end{center}


\newpage
\null

\vfill

\begin{center}
\begin{tabular}[t]{ccc}
............................................. & \hspace*{100pt} & .............................................\\
podpis promotora & \hspace*{100pt} & podpis autora
\end{tabular}
\end{center}


\end{document}